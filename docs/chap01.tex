%%
% 引言或背景
% 引言是论文正文的开端,应包括毕业论文选题的背景、目的和意义;对国内外研究现状和相关领域中已有的研究成果的简要评述;介绍本项研究工作研究设想、研究方法或实验设计、理论依据或实验基础;涉及范围和预期结果等。要求言简意赅,注意不要与摘要雷同或成为摘要的注解。
% modifier: 黄俊杰(huangjj27, 349373001dc@gmail.com)
% update date: 2017-04-15
%%

\chapter{绪论}
% 定义,过去的研究和现在的研究,意义
\label{cha:introduction}
\section{选题背景与意义}
\label{sec:background}
% What is the problem
% why is it interesting and important
% Why is it hards, why do naive approaches fails
% why hasn't it been solved before
% what are the key components of my approach and results, also include any specific limitations,do not repeat the abstract
%contribution
% 引言是论文正文的开端,应包括毕业论文选题的背景、目的和意义;对国内外研究现状和相关领域中已有的研究成果的简要评述;介绍本项研究工作研究设想、研究方法或实验设计、理论依据或实验基础;涉及范围和预期结果等。要求言简意赅,注意不要与摘要雷同或成为摘要的注解。

分布式计算框架是一种复杂的平台软件。传统的并行计算范式如信息传递接口(MPI)和分区全局地址空间(PGAS),通常为编程者提供丰富的调用接口和灵活的编程空间。
然而,使用这些编程标准的门槛过高:编程者通常需要自己管理多台机器的状态,特别是内存的分配和使用;编程者还需要使用给定的标准原语,精确地规定多个进程之间的通信和协作方式,
这通常需要大量时间和精力;另外使用这些范式将导致程序和功能的强耦合——编程者很可能不得不重新编写代码来更新程序的逻辑。因此,现代分布式计算框架通常承担了上述硬件管理的角色,
并针对目标任务类型,为用户设计尽可能少、但简单易用的功能接口,来降低集群的使用难度。传统的分布式计算框架,例如Mapreduce和Spark,都是面向大规模数据分析而设计的。然而,近些年
以强化学习、复杂工作流等为代表的复杂计算需求,需要更灵活的框架支持。加州大学伯克利分校RISELab实验室提出的Ray和芝加哥大学Globus实验室提出的Parsl是其中两个典型。这些框架
吸取了云计算领域“函数即服务(Function as a Service, FaaS)”的设计思想,能够细粒度地以函数为单位将任务调度到集群上执行,同时依然对用户隐藏绝大部分实现细节。Ray通过分布式内存对象存储Plasma,
实现了框架完全自主的集群内存管理,使用户可以专注于实现功能逻辑。Ray在保持易用性的前提下,极大地提升了计算框架的灵活性,用户只需要修改几行代码就能将单进程程序扩展到整个集群。

高性能集群、或超算集群,是一种以高端处理器、计算卡、高性能网络、大容量存储为核心硬件的计算集群。随着大数据应用的丰富、超大规模人工智能模型的出现,高性能集群和超级计算机正在变得愈发重要。
首先,高性能计算机拥有普通机器不能比拟的计算能力,主要表现为高端的多核处理器和计算卡。而随着大模型时代的到来,应用对集群网络的需求逐渐提高,高性能网络所表现出的高带宽、低延迟等特性
也逐渐受到了更多的关注。在超算集群中,网络性能上的优势很大程度上来自于对远程直接内存访问(RDMA)机制的支持。然而,要利用RDMA,用户必须在应用中实现基于RDMA的通信机制,而不能依赖于操作系统内核提供的系统调用。
因此大多数基于套接字(Socket)的网络应用都不能在高性能集群中获得显著的性能提升。分布式计算框架Ray并不例外,其分布式内存存储Plasma目前仅有对传统TCP/IP协议的支持,因而在超算集群上无法发挥出应有的网络性能,从而进一步影响Ray的总体性能。

当前,超算集群普遍使用的是英伟达(Nvidia)公司的Infiniband高速网络。对于使用Socket通信的网络程序,该架构通过“基于Infiniband的互联网协议(IPoIB)”实现支持。值得注意的是,这是一种非原生支持,
已经有多个工作表明,其网络性能和直接使用RDMA技术相比具有明显差距。因此,本研究的目的即:我们是否能为分布式内存存储Plasma,提出并实现一种支持RDMA机制的内存通信协议,从而让Plasma乃至整个Ray框架在
现代超算集群上获得更好的性能?从超算研究的趋势来说,应用软件和先进超算硬件之间的隔阂,正在逐渐成为大家关注的热点。随着超级计算机和云计算两个领域的融合,会有越来越多的软件运行在高性能集群中。
而它们通常没有针对高性能硬件提供支持——通过提供软件对高性能硬件的支持,我们能够将相当多应用的性能提升到全新的水平。

\section{国内外研究现状和相关工作}
\label{sec:related_work}

\subsection{远程直接内存访问技术(RDMA)}



\subsection{基于RDMA技术的内存系统}

\subsubsection{高并发内存系统}

\subsubsection{分布式内存系统}

\section{论文主要研究内容}

本工作研究的是为分布式内存存储Plasma,提供一种原生支持RDMA技术的通信机制,从而使其在现代超算集群上获得更优的数据传输性能。

这一工作存在以下挑战:

\begin{enumerate}
	\item 目前RDMA编程仍然是极为“小众”的技术,如何能基于有限的资料和研究实现高性能的网络通信机制。
	\item 如何在尽可能不破坏项目整体结构的情况下,为Plasma提供原生RDMA通信机制。这要求优化后的程序可以无缝实现两种机制的兼容。
	\item 针对Ray框架中可能出现的大小不一的数据,如何实现该机制使得Plasma能够在尽可能多的范围内都能获得最优的网络性能。
\end{enumerate}

\section{论文结构与章节安排}
\label{sec:arrangement}

本文的剩余章节共分为五章,这些章节的内容安排如下:

第二章为本科生毕业论文写作与印制规范。本章节就学校的规范,逐点进行描述,并给出来了相关例子说明本模板在格式上的正确性。

第三章为本模板的使用说明。

第四章为可用的\LaTeX 的代码段方便大家进行编辑。

第五、六章是本文的最后两章,作为空白章节例子。

