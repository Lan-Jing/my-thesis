%%
% 致谢
% 谢辞应以简短的文字对课题研究与论文撰写过程中曾直接给予帮助的人员(例如指导教师、答疑教师及其他人员)表示对自己的谢意,这不仅是一种礼貌,也是对他人劳动的尊重,是治学者应当遵循的学术规范。内容限一页。
% modifier: 黄俊杰
% update date: 2017-04-15
%%

\chapter{致谢}

本科四年的时光匆匆而逝。尽管这四年的学习、生活并不总是一帆风顺,但在大学中我建立了成熟的世界观、学到了宝贵的知识,并且获得了
继续探索未知的兴趣、思维和方法——借助毕业设计的过程,这些收获最终凝聚在了本文中。因此,我首先要感谢本文的指导老师肖侬教授。他对本研究
的研究方向、研究内容和论文写作等方面都有着非常深入的见解。在肖老师的辅导下,我才能顺利地完成本次设计,并将其总结为论文。

在四年求学的道路上,还有很多老师、同学给予了我帮助。虽不能一一列举,但我非常感谢他们。我要特别感谢超算中心的陈志广老师、黄聃老师,他们用心地指导着超算队的比赛和日常运营;黄聃老师
还在研究上给予了我很多指导,让我产生了进一步研究的兴趣和动力。此外,我要感谢中山大学超算队的学长和队员们,超算队提供给我学习前沿知识、参与系统领域竞赛的机会,
丰富了我的本科生涯。最后,我要感谢我的家人们,他们的默默支持是我前进的最大动力。

\vskip 108pt
\begin{flushright}
	兰靖\makebox[1cm]{} \\
	\today
\end{flushright}

